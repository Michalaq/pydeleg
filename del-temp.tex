\documentclass{article}
\usepackage[margin=2cm]{geometry}
\usepackage{graphicx}
\usepackage{polski}
\usepackage[utf8]{inputenc}
\usepackage{colortbl}
\hyphenpenalty=10000

\definecolor{Gray}{gray}{0.85}
\newcolumntype{a}{>{\columncolor{Gray}}c}

\begin{document}

\title{\textbf{Rachunek kosztów podróży nr \VAR{tripno} z dnia \VAR{startdate} r.}}
\author{dla \VAR{name} \\ do \VAR{destcity}}
\date{}
\maketitle

\subsection*{}
\begin{center}
        na czas od \VAR{startdate} do \VAR{enddate}, w celu: wdrożenie systemu u klienta docelowego
\end{center}

\section{Środki transportu:}

\begin{center}
 \begin{tabular}{| c |  c | c |  c |  c |  c |}
%% for means in [["pociąg", "autbous", "służbowy samochód"], ["samolot", "samochód niefirmowy", "inny .........."]]
  \hline
%% set counter = 0
%% for x in means
  %% if x not in ["pociąg", "samolot"]
    &
  %% endif
  %% if transportmean == x
    X
  %% else
    
  %%endif
  & \VAR{x}
%% endfor
  \\
  \hline
%% endfor
\end{tabular}
\end{center}

\section{Rachunek kosztów podróży}
\begin{center}
 \begin{tabular}{|| c | c | c | c | c | c | p{3cm} | p{2cm} ||} 
  \hline
  \rowcolor{Gray}
  \multicolumn{3}{||c|}{\textbf{WYJAZD}}&\multicolumn{3}{|c|}{\textbf{PRZYJAZD}}& \textbf{Środek} & \textbf{Koszty} \\
  \hline
  \textbf{Miejscowość} & \textbf{Data} & \textbf{Godzina} & \textbf{Miejscowość} & \textbf{Data} & \textbf{Godzina} & \multicolumn{1}{a|}{} & \multicolumn{1}{a||}{} \\
%% for fromcity, fromdate, fromhour, tocity, todate, tohour, mean in trips
  \hline
  \VAR{fromcity} & \VAR{fromdate} & \VAR{fromhour} & \VAR{tocity} & \VAR{todate} & \VAR{tohour} & \VAR{mean} & \\
%% endfor
  \hline
  \multicolumn{6}{||c|}{} & Diety & \VAR{dietval} \VAR{dietcur} \\ 
  \cline{7-8}
  \multicolumn{6}{||c|}{} & Noclegi wg rachunków & \\
  \cline{7-8}
  \multicolumn{6}{||c|}{} & Inne wydatki wg załączników & \\
  \cline{7-8}
  \multicolumn{6}{||c|}{} & \multicolumn{1}{a|}{RAZEM} & \multicolumn{1}{a||}{\VAR{dietval} \VAR{dietcur}} \\
  \hline
  \multicolumn{8}{||l||}{\footnotesize{\textit{Słownie: \VAR{dietvalword}}}} \\
  \hline
\end{tabular}
\end{center}


\subsection{Załączone dowody}
\textbf{Liczba załączonych dowodów:} 0

\subsection*{}

\hfill \begin{tabular}{p{1cm} c p{1cm}} & \VAR{signature} & \\ \hline  & \footnotesize{\textit{Podpis przedsiębiorcy}} & \\ \end{tabular} \\
\\

\end{document}
